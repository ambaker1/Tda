\chapter*{Introduction to Tda}
\addcontentsline{toc}{chapter}{Introduction to Tda}  
Tda version \version\ was written for OpenSees 3.3.0 and Tcl 8.6.10. 
\clearpage
\section{Notation}
This manual is for Tcl-based commands, and notation used was based on the notation used in the original OpenSees documentation \cite{mazzoni_opensees_2006}. 
In this notation style,  the prefix \texttt{\$} is used to denote an input variable, and all other words are literal strings.
Option keywords are typically denoted with the prefix \texttt{-}, and all optional inputs are denoted by enclosing in \texttt{<>} braces.
Finally, an arbitrary number of arguments are denoted by ``1 2 ...'' notation, (e.g. \texttt{\$arg1 \$arg2 ...}), unless if the arguments must be paired, in which case it will use a ``key value ...'' notation.
Below is an example of the notation used for commands in this manual.
\begin{syntax}
command \$foo <-bar> <\$key \$value ...>
\end{syntax}
\begin{args}
\$foo & Required variable input ``foo''. \\
-bar & Optional keyword ``-bar''. \\
\$key \$value ... & Optional paired list (arbitrary number of pairs).
\end{args}
\clearpage
\section{Structure of Tda}
Tda is organized into modules, each contained within a unique namespace and package name, prefixed with \texttt{tda}, the parent namespace/package. 
Versioning follows semantic versioning rules, so one should expect backwards compatibility as long as major version numbers stay the same.
\subsection{Loading Tda Modules}
Loading the main \texttt{tda} package loads all modules listed in Table \ref{tbl:module_versions}.
Alternatively, modules can be individually loaded by specifying the module package name.
Note that dependencies are min-bounded within each module and for the parent package \texttt{tda}, so requiring an exact version of a package does not load the exact versions of its dependencies, but rather the most recent compatible versions.
To load an exact version of Tda, one must load the modules separately, in order of dependence.
\begin{syntax}
package require tda <\$version> \\
package require <-exact> tda::\$module <\$version>
\end{syntax}
\begin{args}
-exact & Option to require an exact version (must also include \$version). \\
\$module & Tda module to require (if just tda, loads all modules). \\
\$version & Specify minimum version number. Default highest stable version.
\end{args}

\clearpage
\subsection{Importing Tda Commands}
Public procedures for each module are imported into the parent namespace \texttt{::tda}, and also added to the \texttt{::tda} namespace export list, for easy import. 
To load all Tda commands, simply import all commands in the \texttt{::tda} export list.
\begin{syntax}
namespace import tda::*
\end{syntax}
To load a specific module, simply import all commands in the module's export list.
\begin{syntax}
namespace import tda::\$module::*
\end{syntax}
\begin{args}
\$module & Tda module to import (if just \texttt{tda::*}, loads all modules). 
\end{args}
To load a specific command, simply specify the command with the \texttt{tda} prefix.
\begin{syntax}
namespace import tda::\$command
\end{syntax}
\begin{args}
\$command & Specific command to import.
\end{args}

\begin{example}{Loading and importing Tda commands}
\begin{lstlisting}
package require tda
puts [tda::range 5]
namespace import tda::*
puts [range 5]
\end{lstlisting}
\tcblower
\begin{lstlisting}
0 1 2 3 4
0 1 2 3 4
\end{lstlisting}
\end{example}

\clearpage
\subsection{Object Oriented Tcl}
Some features in Tda follow an object-oriented paradigm, using the the built-in TclOO package.
In those cases, a ``class'' will be provided that acts as a template for creating ``objects'', or commands that are linked to unique namespaces and have subcommands, or ``methods'' that allow for access and modification of variables in the object's namespace.
Since the TclOO package is utilized, all Tda classes have standard methods ``new'' and ``create'', and all Tda objects have the standard method ``destroy''.
Additionally, as TclOO is standard to Tcl, class and object introspection using the \textit{info} command can be used to dive into the structure of the class (using its fully declared name) and its objects.

To demonstrate TclOO basics, see the example below of a fictitious class named ``foo''.
\begin{example}{TclOO basics}
\begin{lstlisting}
# Create objects from a class named 'foo'
set bar1 [foo new]; # Creates object with auto-generated name, storing in variable 'bar1'
foo create bar2; # Creates object with explicit command name 'bar2'
puts [info class instances foo]; # Display all instances of 'foo'
$bar1 destroy; # Destroys object stored in variable 'bar1'
bar2 destroy; # Destroys object 'bar2'
\end{lstlisting}
\tcblower
\begin{lstlisting}
::oo::Obj12 ::bar2
\end{lstlisting}
\end{example}

\cleartooddpage[\thispagestyle{empty}]
\section{Copyright and Disclaimer}

Copyright (c) 2023, Alexander Baker \\
All rights reserved.

Redistribution and use in source and binary forms, with or without
modification, are permitted provided that the following conditions are met:

1. Redistributions of source code must retain the above copyright notice, this
   list of conditions and the following disclaimer.

2. Redistributions in binary form must reproduce the above copyright notice,
   this list of conditions and the following disclaimer in the documentation
   and/or other materials provided with the distribution.

3. Neither the name of the copyright holder nor the names of its
   contributors may be used to endorse or promote products derived from
   this software without specific prior written permission.

THIS SOFTWARE IS PROVIDED BY THE COPYRIGHT HOLDERS AND CONTRIBUTORS "AS IS"
AND ANY EXPRESS OR IMPLIED WARRANTIES, INCLUDING, BUT NOT LIMITED TO, THE
IMPLIED WARRANTIES OF MERCHANTABILITY AND FITNESS FOR A PARTICULAR PURPOSE ARE
DISCLAIMED. IN NO EVENT SHALL THE COPYRIGHT HOLDER OR CONTRIBUTORS BE LIABLE
FOR ANY DIRECT, INDIRECT, INCIDENTAL, SPECIAL, EXEMPLARY, OR CONSEQUENTIAL
DAMAGES (INCLUDING, BUT NOT LIMITED TO, PROCUREMENT OF SUBSTITUTE GOODS OR
SERVICES; LOSS OF USE, DATA, OR PROFITS; OR BUSINESS INTERRUPTION) HOWEVER
CAUSED AND ON ANY THEORY OF LIABILITY, WHETHER IN CONTRACT, STRICT LIABILITY,
OR TORT (INCLUDING NEGLIGENCE OR OTHERWISE) ARISING IN ANY WAY OUT OF THE USE
OF THIS SOFTWARE, EVEN IF ADVISED OF THE POSSIBILITY OF SUCH DAMAGE.

\chapter*{Introduction to Tda}
\addcontentsline{toc}{chapter}{Introduction to Tda}  
Tda (pronounced "ta-da!"), which stands for ``Tcl Data Analysis'', adds features such as N-dimensional arrays and tabular data structures to Tcl. 
Tda was originally developed for OpenSees, an open-source scripting-based finite-element analysis software, specializing in earthquake engineering simulation \cite{mckenna_nonlinear_2010}, but it is general enough for any Tcl application. 

Tda version \version\ contains the following sub-packages and their respective versions:

\begin{tabular}{lll}
Package & Version & Description \\
\hline
 tda::ndlist & \version[ndlist] & \nameref{ndlist}\\
 tda::tbl & \version[tbl] & \nameref{tbl}\\
 tda::io & \version[io] & \nameref{io}\\
 tda::vis & \version[vis] & \nameref{vis}
\end{tabular}

\clearpage
\section{Notation}
This manual is for Tcl commands, and the notation style is as follows:
\begin{itemize}
\item The prefix \texttt{\$} is used to denote an input variable, and all other words are literal strings.
\item Option keywords are typically denoted with the prefix \texttt{-}, and all optional inputs are denoted by enclosing in \texttt{<>} braces.
\item An arbitrary number of arguments is denoted by ``1 2 ...'' notation, (e.g. \texttt{\$arg1 \$arg2 ...}), unless if the arguments must be paired, in which case it will use a ``key value ...'' notation.
\end{itemize}
Below is an example of the notation used for commands in this manual.
\begin{syntax}
command \$foo <-bar> <\$key \$value ...>
\end{syntax}
\begin{args}
\$foo & Required variable input ``foo''. \\
-bar & Optional keyword ``-bar''. \\
\$key \$value ... & Optional paired list (arbitrary number of pairs).
\end{args}

\clearpage
\section{Loading and Importing Tda Commands}
Tda is organized into modules, each contained within a unique namespace and package name, prefixed with \textit{tda}, the parent namespace/package. 
Loading the main \textit{tda} package using  \textit{package require} loads all the modules.
Alternatively, modules can be individually loaded by specifying the module package names.
\begin{syntax}
package require tda <\$version> \\
package require <-exact> tda::\$module <\$version>
\end{syntax}
\begin{args}
-exact & Option to require an exact version (must also include \$version). \\
\$module & Specific Tda module to require. \\
\$version & Specify minimum version number. Default highest stable version.
\end{args}
When Tda modules are loaded with \textit{package require}, procedures are created within the modules' respective namespaces. These commands can then be accessed with their fully-qualified names (such as \textit{tda::ndlist::range}), or the commands can be imported with \textit{namespace import}, as shown below.
\begin{example}{Loading and importing Tda}
\begin{lstlisting}
package require tda
namespace import tda::*
puts [range 5]
\end{lstlisting}
\tcblower
\begin{lstlisting}
0 1 2 3 4
\end{lstlisting}
\end{example}
\clearpage
\section{Object Oriented Tcl}
Some features in Tda (such as \textit{tda::tbl} tables) follow an object-oriented paradigm, using the built-in ``TclOO'' package. 
Additionally, all Tda widgets are object oriented, using the framework provided by the required package ``wob'' (\hyperlink{https://github.com/ambaker1/wob}{https://github.com/ambaker1/wob}).

In TclOO, a ``class'' command acts as a template for creating ``objects'', or commands that are linked to unique namespaces and have subcommands, or ``methods'' that allow for access and modification of variables in the object's namespace.
Since the TclOO package is utilized, all Tda classes have standard methods ``new'' and ``create'', and all Tda objects have the standard method ``destroy''.
Additionally, as TclOO is standard to Tcl, class and object introspection using the \textit{info} command can be used to dive into the structure of the class (using its fully declared name) and its objects.

To demonstrate TclOO basics, see the example below of a fictitious class named ``foo''.
\begin{example}{TclOO basics}
\begin{lstlisting}
# Create objects from a class named 'foo'
set bar1 [foo new]; # Creates object with auto-generated name, storing in variable 'bar1'
foo create bar2; # Creates object with explicit command name 'bar2'
puts [info class instances foo]; # Display all instances of 'foo'
$bar1 destroy; # Destroys object stored in variable 'bar1'
bar2 destroy; # Destroys object 'bar2'
\end{lstlisting}
\tcblower
\begin{lstlisting}
::oo::Obj12 ::bar2
\end{lstlisting}
\end{example}

\cleartooddpage[\thispagestyle{empty}]
\section*{Copyright and Disclaimer}

Copyright (c) 2023, Alexander Baker \\
All rights reserved.

Redistribution and use in source and binary forms, with or without
modification, are permitted provided that the following conditions are met:

1. Redistributions of source code must retain the above copyright notice, this
   list of conditions and the following disclaimer.

2. Redistributions in binary form must reproduce the above copyright notice,
   this list of conditions and the following disclaimer in the documentation
   and/or other materials provided with the distribution.

3. Neither the name of the copyright holder nor the names of its
   contributors may be used to endorse or promote products derived from
   this software without specific prior written permission.

THIS SOFTWARE IS PROVIDED BY THE COPYRIGHT HOLDERS AND CONTRIBUTORS "AS IS"
AND ANY EXPRESS OR IMPLIED WARRANTIES, INCLUDING, BUT NOT LIMITED TO, THE
IMPLIED WARRANTIES OF MERCHANTABILITY AND FITNESS FOR A PARTICULAR PURPOSE ARE
DISCLAIMED. IN NO EVENT SHALL THE COPYRIGHT HOLDER OR CONTRIBUTORS BE LIABLE
FOR ANY DIRECT, INDIRECT, INCIDENTAL, SPECIAL, EXEMPLARY, OR CONSEQUENTIAL
DAMAGES (INCLUDING, BUT NOT LIMITED TO, PROCUREMENT OF SUBSTITUTE GOODS OR
SERVICES; LOSS OF USE, DATA, OR PROFITS; OR BUSINESS INTERRUPTION) HOWEVER
CAUSED AND ON ANY THEORY OF LIABILITY, WHETHER IN CONTRACT, STRICT LIABILITY,
OR TORT (INCLUDING NEGLIGENCE OR OTHERWISE) ARISING IN ANY WAY OUT OF THE USE
OF THIS SOFTWARE, EVEN IF ADVISED OF THE POSSIBILITY OF SUCH DAMAGE.
